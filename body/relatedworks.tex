\chapter{Literature Review and Related Work}
\label{chap:relatedworks}

In this chapter, describe other solutions/research that address the
same topic as your project. If you are working on a software project, create a
list of alternative solutions and analyze them in the competitor analysis section.
If you are working on a research project, describe your related work research in
the literature review section.

\section{Competitor Analysis}
\label{section:competitor-analysis}
\begin{enumerate}

    \item \textbf{Competitor Overview}
    \begin{enumerate}
        \item WorkQuest competes with:
        \begin{itemize}
            \item Trello \\ Kanban-based task management.
            \item Asana \\ Workflow organization tool.
            \item Monday.com \\ Customizable work OS.
            \item Habitica \\ Gamified personal habit tracker.
            \item Fukumon \\ Gamified productivity with rewards.
        \end{itemize}
    \end{enumerate}
    
    \item \textbf{Market Research}
    \begin{enumerate}
        \item \textbf{Primary Market}
        \begin{itemize}
            \item Use competitors’ services.
        \end{itemize}
        
        \item \textbf{Secondary Market}
        \begin{itemize}
            \item Examine competitors’ websites.
        \end{itemize}
    \end{enumerate}

    \item \textbf{Product Feature Comparison}

    \medskip
    \noindent\begin{center}
    \begin{tabular}{lcccccc}
    \hline
    Features & WorkQuest & Trello & Asana & Monday.com & Habitica & Fukumon \\
    \hline
    Task Management & Yes & Yes & Yes & Yes & Yes & Yes \\
    Gamification & Yes & No & No & No & Yes & Yes \\
    AI Feedback & Yes & No & No & No & No & No \\
    Collaboration Tools & Yes & Yes & Yes & Yes & No & Yes \\
    Reward System & Yes & No & No & No & Yes & Yes \\
    Customization & Yes & Yes & Yes & Yes & No & Yes \\
    \hline
    \end{tabular}
    \caption{Feature comparison among competitors.}
    \label{tab:feature-comparison}
    \end{center}
    \medskip

\item \textbf{Product Marketing Comparison}
    \begin{enumerate}
        \item \textbf{Social Media \& Advertising}
        \begin{itemize}
            \item Trello, Asana, and Monday.com focus on business users.
            \item Habitica and Fukumon highlight gamified motivation.
            \item WorkQuest focus on team engagement.
        \end{itemize}
        
        \item \textbf{Website \& Brand Voice}
        \begin{itemize}
            \item Trello and Asana promote workflow efficiency.
            \item Habitica and Fukumon market personal growth.
            \item WorkQuest focuses on gamified collaboration.
        \end{itemize}
    \end{enumerate}

    \item \textbf{SWOT Analysis}
    \begin{enumerate}
        \item \textbf{Strengths}
        \begin{itemize}
            \item Unique gamified approach with AI feedback.
            \item Focuses on team collaboration.
            \item Engaging reward-based mechanics.
        \end{itemize}
        
        \item \textbf{Weaknesses}
        \begin{itemize}
            \item New to the market.
            \item Requires user adaptation to gamification.
        \end{itemize}

        \item \textbf{Opportunities}
        \begin{itemize}
            \item Increasing demand for gamified productivity.
            \item Schools and companies may find it useful.
        \end{itemize}

        \item \textbf{Threats}
        \begin{itemize}
            \item Strong competition from established platforms.
            \item Users may hesitate to switch tools.
        \end{itemize}
    \end{enumerate}

    \item \textbf{Market Positioning}
    \begin{enumerate}
        \item WorkQuest stands at the intersection of gamification and team collaboration.
        \item Differentiates from competitors by integrating AI feedback and engagement.
    \end{enumerate}

\end{enumerate}

% \begin{figure}[h]
%     \centering
%     \includegraphics[width=0.5\textwidth]{examples/asana-competitive-landscape.jpg}
%     \caption{Competitive Landscape by Asana}
% \end{figure}

Refer to an article "How to create a competitive analysis (with
examples)" by Asana. You can use the Competitor Landscape (left image) or
Competitor Analysis Framework (right image) for your project.

\section{Literature Review}
\label{section:literature-review}
\begin{enumerate}
    \item \textbf{Introduction} \\
        Task management is a critical component of work productivity, yet many teams struggle with disengagement, procrastination, and inefficiencies in collaboration.
        Traditional task management systems often rely on rigid structures that fail to maintain motivation over time. 
        
        In response to these challenges, gamification has emerged as a promising strategy to enhance engagement and performance. by integrating game-like elements—such as goals, rewards, competition, and collaboration—into workplace settings. Research indicates that gamification can increase work engagement by providing team member with a sense of autonomy, competence, and relatedness, thereby creating a fun and engaging work environment \cite{ncbi:pmc10905147}.
        Additionally, studies have demonstrated practical advantages in several performance metrics resulting from the application of gamification strategies, supported by evidence from real-life case studies. \cite{Employee:Gamification}

        WorkQuest introduces a unique approach to task management by integrating gamification elements—specifically "boss fight" mechanics—and AI performance evaluation. In this system, Users collaborate to complete tasks that weaken and ultimately defeat a boss, adding a layer of strategic engagement. AI-driven analysis provides personalized performance feedback and deadline predictions, ensuring that user receive tailored insights to improve efficiency.
    
    \item \textbf{Gamification in Task Management} \\
        Gamification, defined as the application of game-design elements and principles in non-game contexts, has emerged as a powerful strategy for enhancing productivity and engagement in working environment \cite{ncbi:pmc10905147} \cite{Employee:Gamification}.
        by incorporating goal-setting for direction, challenges to maintain interest, rewards to reinforce success, and feedback to guide improvement. These elements work together to keep participants engaged, boost performance, and align intrinsic and extrinsic motivations. \cite{Game:Reward}
        
        Numerous studies have demonstrated the benefits of gamification in a variety of contexts. Research by Juho Hamari. (2014) \cite{6758978} reviewed the impact of gamification across sectors and found that its use consistently improved user engagement, productivity, and satisfaction.
        The use of feedback loops, competition, and goal-setting were particularly effective in motivating individuals to perform tasks efficiently

        In the context of task management, gamification has been shown to improve productivity, collaboration, and task completion rates. By incorporating game-like elements, transforms routine tasks into more engaging and motivating activities.
    
        Moreover, gamification has been shown to promote teamwork and collaboration. Deterding et al. (2011) \cite{gamification:designElement} emphasized that gamified systems enhance group dynamics by incorporating features like team challenges, leaderboards, and shared rewards. These elements encourage users to collaborate, share knowledge, and work together towards common goals, ultimately improving collective productivity and communication within organizations
    
        In conclusion, gamification has proven to be an effective method for enhancing productivity in the workplace. By incorporating game elements that align with psychological motivations, gamified systems can increase task engagement, improve collaboration, and ultimately drive higher levels of productivity and performance. However, successful implementation of gamification depends on aligning its features with organizational goals and users preferences.
    \item \textbf{Boss fight mechanics for engagement} \\
        
    
    \item AI-Driven Performance Evaluation in collaborate work
    
    \item Reward Systems: Intrinsic vs. Extrinsic Motivation
    
    \item Kanban Boards and Gamified Task Visualization
    
    \item AI-Powered Deadline Prediction and Task Analysis
    
    \item Conclusion
    
\end{enumerate}  